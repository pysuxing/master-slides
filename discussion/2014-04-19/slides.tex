\documentclass[]{beamer}
\mode<presentation>{
  %% \usetheme[compress]{Berlin}
}
%% packages
\usepackage{zhspacing}
\zhspacing
\usepackage{graphics}
\usepackage{listings}
\lstset{
  %% basicstyle=\ttfamily\tiny,
  numbers=left
}
\usepackage{tabularx}
\usepackage{booktabs}
%% meta info
\title{Pointer Analysis by Example}
\subtitle{A Comparation of Anderson's and Steensgaard's Method}
\author[SuXing~pysuxing@gmail.com]{SuXing}
\institute{TOW}
\date{\today}

\setbeamercolor{footcolor}{fg=blue,bg=white} %
\setbeamertemplate{footline}{%
  \leavevmode%
  \hbox{%
    \begin{beamercolorbox}[wd=.126\paperwidth,ht=2.25ex,dp=1ex,right]{footcolor}%      
       \insertframenumber{} / \inserttotalframenumber\hspace*{5ex}
    \end{beamercolorbox}}%
  \vskip0pt%
}

%% slides
\begin{document}
\setlength{\parindent}{0pt}

\frame{\titlepage}
\frame{\tableofcontents}

\section{A Simple Example}
\frame{\tableofcontents[currentsection]}

\begin{frame}
  \frametitle{A Simple Example}
  We will use the program fragment below as our example throughout
  this talk
  \pause\vspace{1em}
  \lstinputlisting[language=C]{listings/example.c}
\end{frame}

\section{Anderson's Analysis}
\frame{\tableofcontents[currentsection]}

\begin{frame}
  \frametitle{Anderson's Analysis}
  \framesubtitle{Step 1}
  \begin{columns}
    \begin{column}{.4\textwidth}
      \lstinputlisting[language=C, firstline=1, lastline=1]{listings/example.c}
      \lstinputlisting[language=C, firstline=2, firstnumber=2]{listings/example.c}
    \end{column}
    \begin{column}{.6\textwidth}
      \includegraphics[width=.6\textwidth]{figures/anderson-1}
    \end{column}
  \end{columns}
\end{frame}

\begin{frame}
  \frametitle{Anderson's Analysis}
  \framesubtitle{Step 2}
  \begin{columns}
    \begin{column}{.4\textwidth}
      \lstinputlisting[language=C, firstline=1, lastline=2]{listings/example.c}
      \lstinputlisting[language=C, firstline=3, firstnumber=3]{listings/example.c}
    \end{column}
    \begin{column}{.6\textwidth}
      \includegraphics[width=.6\textwidth]{figures/anderson-2}
    \end{column}
  \end{columns}
\end{frame}

\begin{frame}
  \frametitle{Anderson's Analysis}
  \framesubtitle{Step 3}
  \begin{columns}
    \begin{column}{.4\textwidth}
      \lstinputlisting[language=C, firstline=1, lastline=3]{listings/example.c}
      \lstinputlisting[language=C, firstline=4, firstnumber=4]{listings/example.c}
    \end{column}
    \begin{column}{.6\textwidth}
      \includegraphics[width=.6\textwidth]{figures/anderson-3}
    \end{column}
  \end{columns}
\end{frame}

\begin{frame}
  \frametitle{Anderson's Analysis}
  \framesubtitle{Step 4}
  \begin{columns}
    \begin{column}{.4\textwidth}
      \lstinputlisting[language=C, firstline=1, lastline=4]{listings/example.c}
      \lstinputlisting[language=C, firstline=5, firstnumber=5]{listings/example.c}
    \end{column}
    \begin{column}{.6\textwidth}
      \includegraphics[width=.6\textwidth]{figures/anderson-4}
    \end{column}
  \end{columns}
\end{frame}

\begin{frame}
  \frametitle{Anderson's Analysis}
  \framesubtitle{Step 5}
  \begin{columns}
    \begin{column}{.4\textwidth}
      \lstinputlisting[language=C, firstline=1, lastline=5]{listings/example.c}
      \lstinputlisting[language=C, firstline=6, firstnumber=6]{listings/example.c}
    \end{column}
    \begin{column}{.6\textwidth}
      \includegraphics[width=.6\textwidth]{figures/anderson-5}
    \end{column}
  \end{columns}
\end{frame}

\section{Steensgaard's Analysis}
\frame{\tableofcontents[currentsection]}

\begin{frame}
  \frametitle{Steensgaard's Analysis}
  Steensgaard's analysis gets the same result as Anderson's for the first 4 assignment
  statements, but behaves differently processing statement 5
\end{frame}

\begin{frame}
  \frametitle{Steensgaard's Analysis}
  \framesubtitle{Step 4}
  \begin{columns}
    \begin{column}{.4\textwidth}
      \lstinputlisting[language=C, firstline=1, lastline=4]{listings/example.c}
      \lstinputlisting[language=C, firstline=5, firstnumber=5]{listings/example.c}
    \end{column}
    \begin{column}{.6\textwidth}
      \includegraphics[width=.6\textwidth]{figures/steensgaard-4}
    \end{column}
  \end{columns}
\end{frame}

\begin{frame}
  \frametitle{Steensgaard's Analysis}
  \framesubtitle{Step 5-A}
  \begin{columns}
    \begin{column}{.4\textwidth}
      \lstinputlisting[language=C, firstline=1, lastline=5]{listings/example.c}
      \lstinputlisting[language=C, firstline=6, firstnumber=6]{listings/example.c}
    \end{column}
    \begin{column}{.6\textwidth}
      \includegraphics[width=.6\textwidth]{figures/steensgaard-5a}
    \end{column}
  \end{columns}
\end{frame}

\begin{frame}
  \frametitle{Steensgaard's Analysis}
  \framesubtitle{Step 5-B}
  \begin{columns}
    \begin{column}{.4\textwidth}
      \lstinputlisting[language=C, firstline=1, lastline=5]{listings/example.c}
      \lstinputlisting[language=C, firstline=6, firstnumber=6]{listings/example.c}
    \end{column}
    \begin{column}{.6\textwidth}
      \includegraphics[width=.6\textwidth]{figures/steensgaard-5b}
    \end{column}
  \end{columns}
\end{frame}

\begin{frame}
  \frametitle{Steensgaard's Analysis}
  \framesubtitle{Step 5-C}
  \begin{columns}
    \begin{column}{.4\textwidth}
      \lstinputlisting[language=C, firstline=1, lastline=5]{listings/example.c}
      \lstinputlisting[language=C, firstline=6, firstnumber=6]{listings/example.c}
    \end{column}
    \begin{column}{.6\textwidth}
      \includegraphics[width=.6\textwidth]{figures/steensgaard-5c}
    \end{column}
  \end{columns}
\end{frame}

\begin{frame}
  \frametitle{The Differences}
  In Steensgaard's analysis, each node has \alert{exactly 1} point-to node
  \begin{itemize}
    \item scalibility increased as each complex assignment ``p=*q'', ``*p=q'' can be processed in $O(1)$ time
      instead of $O(V)$
    \item the tradeoff: precision lose
  \end{itemize}
  \pause
  While Anderson's analysis has an $O(n^3)$ complexity, the overall complexity of
  Steensgaard's is only $O(n\alpha(n))$(almost linear)
\end{frame}

\section{Summary}
\frame{\tableofcontents[currentsection]}

\begin{frame}
  \frametitle{Summary}
  \begin{table}
    \begin{tabular}{|c|c|c|}
      \hline
      Assignment & Anderson's Constraint & Steensgaard's Constraint\\
      \hline
      a = \&b & $a \supseteq \{b\}$ & $a = \{b\}$\\
      a = b & $a \supseteq b$ & $a = b$\\
      a = *b & $a \supseteq *b$ & $a = *b$ \\
      *a = b & $*a \supseteq b$ & $*a = b$\\
      \hline
    \end{tabular}
    \caption{Anderson's v.s. Steensgaard's}
  \end{table}
\end{frame}

\begin{frame}
  \frametitle{References}
  \begin{itemize}
    \item Lecture Notes 7 of CMU course ``Program Analysis'' (15-819O)\\
      http://www.cs.cmu.edu/aldrich/courses/15-819O-13sp/
    \item Pointer-to Analysis in Almost Linear Time, Steensgaard Bjarne,
      Proceedings of the 23rd ACM SIGPLAN-SIGACT symposium on Principles of
      Programming Languages, 1996
  \end{itemize}
\end{frame}

\frame{\centerline{\Huge Q\&A}}

\end{document}
