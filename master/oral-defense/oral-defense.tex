\documentclass{beamer}
\mode<presentation>{
  %% \usetheme[compress]{Berlin}
}
%% packages
\usepackage{zhspacing}
\zhspacing
\usepackage{graphics}
%% meta info
\title{函数式并行程序语言研究}
\author[苏醒~pysuxing@gmail.com]{
\begin{tabular}{ll}
答辩人: & 苏醒 \\
指导教师: & 窦文华~教授
\end{tabular}
}
\institute{计算机所641教研室}
\date{\today}

%% slides
\begin{document}
\setlength{\parindent}{0pt}

\section{Background}

\subsection{Parallel and Heterogeneous Hardware}
\begin{frame}
  \frametitle{Trends of computing hardware}
  Processors are no longer going faster ...

  For higher performance, we are harnessing more computers % FIXME: figure here

  and using more CPUs in a single computer % FIXME: figure here

  and put more cores inside a single CPU % FIXME: figure here

  Even more, we orgnize computer systems using different computing hardware.

  %% FIXME: tianhe, pc, tablet
\end{frame}

\begin{frame}
  \frametitle{Trends of computing hardware (cont.)}
  Computer system is becoming:
  \begin{itemize}
    \item parallelized
    \item heterogeneous
  \end{itemize}
\end{frame}

\subsection{Parallel Software}
\begin{frame}
  \frametitle{Goals of parallel software}
  We expect parallel software techniques to be
  \begin{itemize}
    \item effective\\
      so hardware power is fully exploited
    \item easy-to-use\\
      to program more productively
  \end{itemize}
\end{frame}

\begin{frame}
  \frametitle{How we program on parallel hardware?}
  Traditionally, we use multi-threading % FIXME: code here

  and in HPC, MPI is often the choice % FIXME: code here

  or compiler directives such as OpenMP % FIXME: code here
\end{frame}

\begin{frame}
  \frametitle{Parallel language as an alternative}
  Parallel programming has always been hard cause there are so many hardware details to deal with ...
  \begin{itemize}
    \item memory
    \item threads/processes
    \item communication
  \end{itemize}

  So parallel programming languages are attractive
  \begin{itemize}
    \item parallel syntax structure or parallel funcion primitives
    \item high level of abstraction
    \item details transparent to end users
  \end{itemize}
\end{frame}

\section{Research}
\begin{frame}                   % FIXME: consider font size in this frame
  Rat: A {\Large FUNCTIONAL PARALLEL} Programming Language
\end{frame}

\begin{frame}
  \frametitle{Main contributions}
  \begin{enumerate}
    \item a functional language design
    \item a flexible programming model based on vector primitives
    \item automatical parallelizing technique based on data-flow
    \item a runtime system exploiting CPU/GPU heterogeneous system
  \end{enumerate}
\end{frame}

\subsection{Functional Design}
\frame{\Large \#1 functional design of Rat}

\begin{frame}
  \frametitle{``first-class citizen'' functions}
  Rat has ``first-class citizen'' functions, that is, functions can be
  \begin{itemize}
    \item arguments of other functions
    \item return value of other functions
    \item anonymous(lambda expression)
  \end{itemize}

  \emph{FUNCTION = DATA!}             % FIXME: use font color instead of emph here
\end{frame}

\begin{frame}
  \frametitle{pure functional}
  AKA ``side-effect free''. 
  \begin{definition}
    A function is side-effect free if it's result depends only on its arguments.
  \end{definition}
  \begin{itemize}
    \item pure functions \\\texttt{sqrt, pow, exp ...}
    \item side-effect functions \\\texttt{printf, readline ...}
  \end{itemize}
  %% \pause
  %% lambda演算的理论保证了一个表达式的值与各个子表达式的求值顺序无关。
\end{frame}

\begin{frame}
  \frametitle{curried functions}
  Rat's functions are curried
  \begin{definition}
    A function is curried if function parameters can be partial instantiated.
  \end{definition}
  An other perspect: a n-ary function is regarded as a unary function whose return value
  is a (n-1)-ary function.
\end{frame}

\begin{frame}
  \frametitle{immutable object}
  Given a value at definition time, an Rat object will hold
  this value for its whole lifetime

  \emph{No assignment!}         % FIXME, font color
  %% \pause
  %% 恒值对象与纯函数特性可以大大简化程序语义的分析。
\end{frame}

\begin{frame}
  \frametitle{Why functional?}
  Advantages of functional languages
  \begin{itemize}
    \item high level of abstraction
    \item simple semantics
    \item pure functional character makes parallelization easier
  \end{itemize}

  Other programming paradims
  \begin{itemize}
    \item imperative languages\\
      low abstraction level, hardware
      抽象级别低,细节隐藏能力差,强迫编程者以计算机的逻辑思考问题
    \item object-oriented languages\\
      unsuitable for numeric calculating
  \end{itemize}
\end{frame}

\subsection{Programming Model}
\subsection{Automatical Parallelization}
\subsection{Runtime System}

\end{document}
